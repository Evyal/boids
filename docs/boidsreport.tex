\documentclass[10pt,a4paper]{article}

\usepackage[utf8]{inputenc}
\usepackage{graphicx}

\usepackage[margin=19mm]{geometry}
\parskip 4.0pt  % Sets spacing between paragraphs.
% \renewcommand{\baselinestretch}{1.5}  % Uncomment for 1.5 spacing between lines.
\parindent 8.0pt  % Sets leading space for paragraphs.
\usepackage[font=sf]{caption} % Changes font of captions.

\usepackage{amsmath}
\usepackage{mathtools}
\usepackage{esint}
\usepackage{amssymb}
\usepackage{amsfonts}
\usepackage{multicol}
\usepackage{tabularx}
\usepackage{booktabs}
\usepackage{url} % Links
\usepackage[colorlinks=true, linkcolor=blue, urlcolor=black, citecolor=green]{hyperref} % Links in words

\usepackage{subfiles}

\usepackage{listings}
\usepackage{xcolor}



\definecolor{codegreen}{rgb}{0,0.6,0}
\definecolor{codegray}{rgb}{0.5,0.5,0.5}
\definecolor{codepurple}{rgb}{0.58,0,0.82}
\definecolor{backcolour}{rgb}{0.95,0.95,0.92}

\lstdefinestyle{mystyle}{
    backgroundcolor=\color{backcolour},   
    commentstyle=\color{codegreen},
    keywordstyle=\color{magenta},
    numberstyle=\tiny\color{codegray},
    stringstyle=\color{codepurple},
    basicstyle=\ttfamily\footnotesize,
    breakatwhitespace=false,         
    breaklines=true,                 
    captionpos=b,                    
    keepspaces=true,                 
    numbers=left,                    
    numbersep=5pt,                  
    showspaces=false,                
    showstringspaces=false,
    showtabs=false,                  
    tabsize=2
}

\lstset{style=mystyle, language= C++}

\title{Relazione Progetto Boids}
\author{Francesco Bartoli}
\date{}

\begin{document}

\setlength{\parindent}{0pt}

\maketitle

\section{Introduzione}
\subsection{Scopo}
Il programma ha come obiettivo quello di simulare in uno spazio bidimensionale il comportamento di stormi di uccelli in volo, che verranno indicati con il nome di \textit{boids}. 

\subsection{Installazione}

Le istruzioni su come compilare, testare, eseguire sono presentate nel README del progetto, che riporto qui sotto:

\par\noindent\rule{\textwidth}{0.4pt}

Build instructions are for \textbf{Ubuntu 22.04}.

\subsubsection{Prerequisites}

\href{https://github.com/SFML/SFML}{SFML} (2.5): Library for graphic representation. \\
\href{https://github.com/texus/TGUI}{TGUI} (1.0): Library for graphic interface.

% \subsubsection{SFML and TGUI Installation}

% Install SFML:

% \begin{lstlisting}
% sudo apt-get install libsfml-dev
% \end{lstlisting}

% \vspace{5pt}
% \hrule
% \vspace{5pt}

% Install TGUI:

% \begin{lstlisting}
% sudo add-apt-repository ppa:texus/tgui
% sudo apt update
% sudo apt install libtgui-1.0-dev
% \end{lstlisting}

\subsubsection{Clone the Repository}

\begin{lstlisting}
git clone https://github.com/Evyal/boids.git
\end{lstlisting}

\subsubsection{Build the Project}

    
\title{\textbf{Create a build directory}}

\begin{lstlisting}
mkdir build
cd build
\end{lstlisting}

\title{\textbf{Configure CMake in Release mode}}

\begin{lstlisting}
cmake .. -DCMAKE_BUILD_TYPE=Release
\end{lstlisting}

\title{\textbf{Build the project}} 

\begin{lstlisting}
cmake --build .
\end{lstlisting}

\subsubsection{Running the program} 

\begin{lstlisting}
./boids
\end{lstlisting}

\newpage

\section{Struttura del programma}
Segue una breve descrizione sintetica delle principali scelte progettuali e implementative del programma.

\subsection{Regole di volo}

I \textit{boids} si seguono delle regole di volo, che ne determinano il comportamento. Ad ogni istante, il programma modifica le velocità e le posizioni dei \textit{boids} attraverso le seguenti formule:

\begin{equation*}
    \vec{v}_{bi} = \vec{v}_{bi} + \vec{v}_S + \vec{v}_A + \vec{v}_C + \vec{v}_R
\end{equation*}

\begin{equation*}
    \vec{x}_{bi} = \vec{x}_{bi} + \vec{v}_{bi} \Delta t
\end{equation*}

Dove $\vec{v}_S$, $\vec{v}_A$, $\vec{v}_C$, e $\vec{v}_R$ sono rispettivamente:

\subsubsection{Separazione}

\begin{equation*}
    \vec{v}_1 = -s \sum_{j \neq i} (\vec{x}_{b_j} - \vec{x}_{b_i}) \quad \text{se} \quad \left| \vec{x}_{b_i} - \vec{x}_{b_j} \right| < d_s
\end{equation*}

\subsubsection{Allineamento}

\begin{equation*}
    \vec{v}_2 = a \left( \frac{1}{n-1} \sum_{j \neq i} \vec{v}_{b_j} - \vec{v}_{b_i} \right) \quad \text{se} \quad \left| \vec{x}_{b_i} - \vec{x}_{b_j} \right| < i
\end{equation*}

\subsubsection{Coesione}

\begin{equation*}
    \vec{x}_c = \frac{1}{n-1} \sum_{j \neq i} \vec{x}_{b_j} \quad \text{se} \quad \left| \vec{x}_{b_i} - \vec{x}_{b_j} \right| < i
\end{equation*}

\begin{equation*}
    \vec{v}_3 = c (\vec{x}_{c} - \vec{x}_{b_j})
\end{equation*}

Dove $n-1$ assume valori dipendenti dal numero di boid nel range di interazione, e non è un valore fisso dipendente dal numero di boids nello stormo.

E $s, ds, a, c, i$ sono parametri della simulazione.

\subsubsection{Repulsione}

\begin{equation*}
    \vec{x}_c = \frac{1}{n-1} \sum_{j \neq i} \vec{x}_{b_j} \quad \text{se} \quad \left| \vec{x}_{b_i} - \vec{x}_{b_j} \right| < dr
\end{equation*}

Questa regola determina l'allontanamento tra \textit{boids} di stormi differenti e introduce due nuovi parametri $r, dr$.

\subsubsection{Interazione \textit{On Click}}

\begin{equation*}
    \vec{v}_1 = \pm p \sum_{j} (\vec{x}_{b_j} - \vec{x}) \quad \text{se} \quad \left| \vec{x}_{b_j} - \vec{x} \right| < i
\end{equation*}

Questa regola permette all'utente di interagire con i boids, e introduce l'ultimo parametro che determina il comportamento dei boids. Il $\pm$ è dovuto al fatto che questa interazione può essere sia attrattiva che repulsiva mentre $\vec{x}$ è il punto in cui l'utente ha cliccato.

\newpage

\subsection{File di implementazione}

Tutti i file menzionati come .cpp hanno un corrispettivo header.

\subsubsection{constants.hpp}

Namespace che contiene valori come limiti di velocità o posizione per i \textit{boids}, parametri di interazione di default, o ulteriori valori per l'inizializzazione degli elementi dell'interfaccia grafica.

\subsubsection{structs.hpp}

Struct per impacchettare dei valori usati per inizializzare bottoni o altri elementi di interfaccia grafica

\subsubsection{boid.cpp}

File di implementazione della classe Boid e funzioni ausiliare per gestirne il comportamento

\subsubsection{flock.cpp}

File di implementazione della classe Flock che determina la struttura collettiva dei \textit{boids} all'interno di uno stormo.

\subsubsection{random.cpp}

File che si occupa della generazione di numeri casuali

\subsubsection{statistics.cpp}

File che si occupa del calcolo delle statistiche restituite a schermo, riguardanti valori medi e deviazioni standard delle posizioni e velocità dei \textit{boids}.

\subsubsection{graphics.cpp}

Breve file che contiene una funzione per la corretta rappresentazione grafica dei \textit{boids}, ed un'altra che costruisce un rettangolo (sf::Rectangle) prendendo come parametro una delle struct definita nel file sopracitato.

\subsubsection{switchbutton.cpp}

La classe Switchbutton introduce la funzionalità di un bottone che può trovarsi in due stati, non fornita da TGUI.

\subsubsection{gui.cpp}

Classe che introduce gli elementi necessari a costruire l'interfaccia grafica del programma, e si occupa di coordinare tutti i file che implementano la logica all'interno di essa. 

\newpage

\section{Interfaccia della simulazione}
%descrizione del formato di input e di output, possibilmente con degli esempi
\newpage

\section{Testing}
%strategia di test per verificare che quanto ottenuto sia ragionevolmente esente da errori

Tutti i file incaricati dell'implementazione di parte della logica del programma hanno un corrispettivo file di testing. Più precisamente, sono presenti i seguenti: testboid.cpp, testflock.cpp, testrandom.cpp e teststatistics.cpp. 

Attraverso i test si è cercato di controllare che le classi, i metodi delle classi e le funzioni introdotte fossero esenti da errori e mostrassero il comportamento atteso. Sono stati eseguiti test in casi semplici per poter stabilire il funzionamento corretto del codice, e anche in alcuni casi particolari quando ritenuto necessario.

Il framework che si è utilizzato per creare le testing unit è doctest.h, il cui file è incluso nella cartella nel progetto (/assets/doctest.h). Questa libreria è in grado di generare autonomamente un main e permette l'esecuzione dei test semplicemente includendo il file sopracitato.

Per potere eseguire i test è necessario trovarsi nella cartella dove vengono prodotti gli eseguibili dei file precedentemente menzionati, seguendo i passaggi elencati sotto. Immaginando di trovarsi nella cartella principale dov'é contenuto il progetto:

\begin{lstlisting}
    cd build
    cd testing
\end{lstlisting}

E digitare il comando corrispondente al test che si vuole eseguire: 

\begin{lstlisting}
    ./testboid
    ./testflock
    ./testrandom
    ./teststatistics
\end{lstlisting}

\newpage

\section{Conclusioni}
%interpretazione dei risultati ottenuti

\subsection{Descrizione dei risultati}




\end{document}