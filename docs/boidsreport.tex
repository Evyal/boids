\documentclass[12pt,a4paper]{article}

\usepackage[utf8]{inputenc}
\usepackage{graphicx}

\usepackage[margin=25mm]{geometry}
\parskip 4.2pt  % Sets spacing between paragraphs.
% \renewcommand{\baselinestretch}{1.5}  % Uncomment for 1.5 spacing between lines.
\parindent 8.4pt  % Sets leading space for paragraphs.
\usepackage[font=sf]{caption} % Changes font of captions.

\usepackage{amsmath}
\usepackage{mathtools}
\usepackage{esint}
\usepackage{amssymb}
\usepackage{amsfonts}
\usepackage{multicol}
\usepackage{tabularx}
\usepackage{booktabs}
\usepackage{url}

\usepackage{subfiles}

\usepackage{listings}
\usepackage{xcolor}

\definecolor{codegreen}{rgb}{0,0.6,0}
\definecolor{codegray}{rgb}{0.5,0.5,0.5}
\definecolor{codepurple}{rgb}{0.58,0,0.82}
\definecolor{backcolour}{rgb}{0.95,0.95,0.92}

\lstdefinestyle{mystyle}{
    backgroundcolor=\color{backcolour},   
    commentstyle=\color{codegreen},
    keywordstyle=\color{magenta},
    numberstyle=\tiny\color{codegray},
    stringstyle=\color{codepurple},
    basicstyle=\ttfamily\footnotesize,
    breakatwhitespace=false,         
    breaklines=true,                 
    captionpos=b,                    
    keepspaces=true,                 
    numbers=left,                    
    numbersep=5pt,                  
    showspaces=false,                
    showstringspaces=false,
    showtabs=false,                  
    tabsize=2
}

\lstset{style=mystyle, language= C++}

\title{Relazione Progetto Boids}
\author{Francesco Bartoli}
\date{}

\begin{document}

\setlength{\parindent}{0pt}

\maketitle

\section{Introduzione}
\subsection{Scopo}
Il programma ha come obiettivo quello di simulare in uno spazio bidimensionale il comportamento di stormi di uccelli in volo, che verranno indicati con il nome di \textit{boids}. 

\subsection{Installazione}

Le istruzioni su come compilare, testare, eseguire sono anche presenti nel README del progetto, che riporto qui sotto:

Build instructions are for **Ubuntu 22.04**.

Ensure you have the following installed:

- [SFML](https://github.com/SFML/SFML) (2.5): Library for graphic representation.
- [TGUI](https://github.com/texus/TGUI) (1.0): Library for graphic interface.

\subsection{SFML and TGUI Installation}

Install SFML:x

\begin{lstlisting}
sudo apt-get install libsfml-dev
\end{lstlisting}

Install TGUI:

\begin{lstlisting}
sudo add-apt-repository ppa:texus/tgui
sudo apt update
sudo apt install libtgui-1.0-dev
\end{lstlisting}

\subsection{Clone the Repository}

\begin{lstlisting}
git clone https://github.com/Evyal/boids.git
\end{lstlisting}

\subsection{Build the Project}

    
\subsubsection{Create a build directory:}

\begin{lstlisting}
mkdir build
cd build
\end{lstlisting}

\subsubsection{Configure CMake in Release mode}

\begin{lstlisting}
cmake .. -DCMAKE_BUILD_TYPE=Release
\end{lstlisting}

\subsubsection{Build the project} 

\begin{lstlisting}
cmake --build .
\end{lstlisting}

\subsection{Running the program} 

\begin{lstlisting}
./boids
\end{lstlisting}

\section{Struttura del programma}
descrizione sintetica delle principali scelte progettuali e implementative

\subsection{Regole di volo}
\subsection{File di implementazione}



\section{Interfaccia della simulazione}
descrizione del formato di input e di output, possibilmente con degli esempi

\section{Testing}
strategia di test per verificare che quanto ottenuto sia ragionevolmente esente da errori

\section{Conlcusioni}
interpretazione dei risultati ottenuti





\end{document}